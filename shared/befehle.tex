% Copyright 2009 Dominik Wagenfuehr <dominik.wagenfuehr@deesaster.org>
% Dieses Dokument unterliegt der Creative-Commons-Lizenz
% "Namensnennung-Weitergabe unter gleichen Bedingungen 3.0 Deutschland"
% [http://creativecommons.org/licenses/by-sa/3.0/de/].

%%%%%%%%%%%%%%%
% Farben
%%%%%%%%%%%%%%%

\definecolor{mittelblau}{rgb}{0.1,0.5,1.0}
\definecolor{orange}{rgb}{1,0.39,0.04}
\definecolor{hellgelb}{rgb}{1.0,1.0,0.9}
\definecolor{hellgrau}{gray}{0.93}
\definecolor{mittelgrau}{gray}{0.85}
\definecolor{dunkelgrau}{gray}{0.35}
\definecolor{url}{rgb}{0,0,0}

% Hauptfarbe für die Seite
\newcommand*{\maincolor}{mittelblau}
\newcommand*{\maintextcolor}{white}

\newcommand*{\SetMainColor}[1]{\renewcommand*{\maincolor}{#1}}
\newcommand*{\SetMainTextColor}[1]{\renewcommand*{\maintextcolor}{#1}}

%%%%%%%%%%%%%%%%%%%%%%%%%%%%%%%%%%
% neue Keyvalues
%%%%%%%%%%%%%%%%%%%%%%%%%%%%%%%%%%

\newlength{\BildImageWidth}      % image width
\newboolean{BildAbsolutePos}     % absolute position over paper width
\newlength{\BildBoxWidth}        % Width of box with image caption
\newcommand*{\BildBoxAlign}{}    % alignment of box

\makeatletter
\define@key{Bild}{width}{\setlength{\BildImageWidth}{#1}}
\define@boolkey{Bild}{absolut}[true]{\setboolean{BildAbsolutePos}{#1}}
\define@key{Bild}{boxwidth}{\setlength{\BildBoxWidth}{#1}}
\define@choicekey{Bild}{boxalign}[\val\ba]{left,right}
{%
    \ifcase\ba\relax
      \renewcommand*{\BildBoxAlign}{}
    \or
      \renewcommand*{\BildBoxAlign}{\hspace*{\fill}}
    \fi
}
\makeatother

%%%%%%%%%%%%%%%%%%%%%%%%%%%%%%%%%%
% Farbige Boxen
%%%%%%%%%%%%%%%%%%%%%%%%%%%%%%%%%%

% Makro für eine farbige Box mit farbigem Text
% Benutzung: \colbox{HINTERGRUNDFARBE}{SCHRIFTFARBE}{TEXT}
% Beispiel: \colbox{black}{white}{Hallo}
\newcommand{\colbox}[3]{\colorbox{#1}{\textcolor{#2}{#3}}}

% Makro für eine farbige Box mit farbigem Rand und farbigem Text
% Benutzung: \fcolbox{RAHMENFARBE}{HINTERGRUNDFARBE}{SCHRIFTFARBE}{TEXT}
% Beispiel: \fcolbox{red}{black}{white}{Hallo}
\newcommand{\fcolbox}[4]{\fcolorbox{#1}{#2}{\textcolor{#3}{#4}}}

%%%%%%%%%%%%%%%%%%%%%%%%%%%%%%%%%%
% Artikellayout-Definitionen
%%%%%%%%%%%%%%%%%%%%%%%%%%%%%%%%%%

% Makro für die Großbuchstaben am Anfang eines Artikels
% Benutzung: \Initial{BUCHSTABE}
% Beispiel: \Initial{M}
\newcommand{\Initial}[1]{%
    \setcounter{DefaultLines}{3} %
    \renewcommand*{\DefaultLoversize}{-0.2} %
    \setlength{\fboxsep}{0.4em}%
    \lettrine[nindent=-0.1mm,findent=0.7em,lraise=0.3]%
    {%
        \colbox{\maincolor}{\maintextcolor}{%
            \begin{minipage}{0.9em}%
            \centering{}%
            #1 %
            \end{minipage}%
        }%
    }%
    {}%
}

% Trenner für Abschnitt
% Benutzung: \Abschnitt{TEXT}
% Beispiel: \Abschnitt{Ubuntu}
\newcommand{\Abschnitt}[1]{%
    \setlength{\fboxsep}{0.2em}%
    \subsection*{\colbox{\maincolor}{\maintextcolor}{\parbox{\linewidth-2\fboxsep-2\fboxrule}{#1}}}%
    \noindent{}%
}

% Darstellung des Autors/der Autoren
% Benutzung: \Autor{NAME}
% Beispiel: \Autor{Max Mustermann}
\newcommand{\Autor}[1]
{%
    \begin{spacing}{1.2}%
        \begin{description}%
            \item[\textbf{Von:}] #1%
        \end{description}%
    \end{spacing}%
}

% Kurze Einleitung zum Text
% Benutzung: \Einleitung{Text}
% Beispiel: \Einleitung{Blah blah blah.}
\newcommand{\Einleitung}[1]
{%
    \begin{spacing}{2}%
        \begin{flushleft}%
            {\huge #1}%
        \end{flushleft}%
    \end{spacing}%
    \vfill{}%
    \columnbreak{}%
}

% Artikelumgebung
% Benutzung: \begin{Artikel}{PDFMARKE}{ÜBERSCHRIFT}{AUTOR/EN}{EINLEITUNG}
%            ...
%            \end{Artikel}
% Beispiel: \begin{Artikel}{2009_10_linux}
%           {Linux auf dem Vormarsch}
%           {Max Mustermann}
%           {Blah blah blah.}
%           ...
%           \end{Artikel}
\newenvironment{Artikel}[4]
{%
    \pdfbookmark[2]{#2}{#1}
    \setcounter{linkcounter}{1}
    \begin{multicols*}{4}[\section*{#2}]
        \begin{spacing}{1.05}
            \Autor{#3}
            \vskip-3em{}%
            \Einleitung{#4}
}
{%
        \end{spacing}
    \end{multicols*}
}

% Bildunterschrift für Bilder
% \Autoreninfo[BREITE]{AUTOR/EN}{TEXT}
\newcommand{\Autoreninfo}[3][1.0]{%
    \setlength{\fboxsep}{0pt}%
    \vskip1em{}%
    \noindent\colbox{hellgrau}{black}{%
        \parbox{#1\linewidth-2\fboxsep}{%
            \setlength{\fboxsep}{4pt}%
            \colbox{dunkelgrau}{white}{%
                \parbox{\linewidth-2\fboxsep}{{\Large #2}}%
            }\\[0.3em]%
            \colbox{hellgrau}{black}{%
                \parbox{\linewidth-2\fboxsep}{#3}%
            }\\[0.3em]%
        }%
    }%
} 

%%%%%%%%%%%%%%%%%%%%%%%%%%%%%%%%%
% Bilder
%%%%%%%%%%%%%%%%%%%%%%%%%%%%%%%%%

% Bildunterschrift für Bilder
% \Bildunterschrift{BREITE}{BILDUNTERSCHRIFT1}{BILDUNTERSCHRIFT2}
\newcommand{\Bildunterschrift}[3]
{%
    \setlength{\fboxsep}{5pt}%
    \setlength{\fboxrule}{0.8pt}%
    \noindent{}%
    \fcolbox{black}{dunkelgrau}{white}{%
        \parbox{#1-2\fboxsep-2\fboxrule}{%
            \begin{flushleft}%
                \vskip-5pt{}%
                \textbf{\MakeUppercase{#2}} | #3%
            \end{flushleft}%
            \vskip-5pt{}%
        }%
    }%
}

% Verarbeitet die optionalen Parameter eines Bildes (nur für interne Verwendung)
\newcommand*{\CalcBildValues}[1]
{%
    \presetkeys{Bild}{width=\linewidth,absolut=false,boxwidth=0pt,boxalign=left}{}%
    \setkeys{Bild}{#1}%
    \ifthenelse{\boolean{BildAbsolutePos}}
    {%
        \setlength{\BildImageWidth}{\paperwidth}%
    }{}%
    \ifdim\BildBoxWidth=0pt{}%
        \setlength{\BildBoxWidth}{\BildImageWidth}%
    \fi{}%
}

% Einbinden von Bildern mit Bildunterschrift (nur für interne Verwendung)
% \Bild[BREITE]{BILD.PNG}{BILDUNTERSCHRIFT1}{BILDUNTERSCHRIFT2}
\newcommand{\BildInternal}[3]
{%
    \includegraphics[width=\BildImageWidth]{#1}\\[-2px]
}

% Definition neuer Länge für Abstand bei Bild.
\newlength{\VSkipHeight}

% Einbinden von Bildern mit Bildunterschrift
% \Bild[OPTIONEN]{BILD.PNG}{BILDUNTERSCHRIFT1}{BILDUNTERSCHRIFT2}
\newcommand{\Bild}[4][]
{%
    \CalcBildValues{#1}%
    \ifthenelse{\boolean{BildAbsolutePos}}
    {%
        \begin{textblock}{1}(0,0)%
    }{}%
    \noindent{}%
    \parbox{\paperwidth}{\BildInternal{#2}{#3}{#4}}%
    \ifthenelse{\boolean{BildAbsolutePos}}
    {%
        \end{textblock}%
        \settoheight{\VSkipHeight}{\BildInternal{#2}{#3}{#4}}%
        \addtolength{\VSkipHeight}{-3em}%
        \vspace*{\VSkipHeight}%
    }{}%
}

% Makro für Absätze (halbe Leerzeilen) bei mehrspaltigen Bildern
% Benutzung: \Einschub{LEERZEILEN}{RECHTS_ODER_LINKS}{BREITE}
% Beispiel: \Einschub{26}{l}{1cm}
\newcommand{\Einschub}[3]
{%
    \begin{wrapfigure}[#1]{#2}{#3}
    \vfill
    \end{wrapfigure}
}

%%%%%%%%%%%%%%%%%%%%%%%%%%%%%%%%%
% Kopf-/Fußleiste
%%%%%%%%%%%%%%%%%%%%%%%%%%%%%%%%%

% Setzt die aktuell Kategorie und Farbthema
\newcommand*{\SetKategorie}[3]{%
    \fancyhead{}%
    \fancyhead[LE]{\PrintHeader{#1}}%
    \pdfbookmark[1]{#1}{#1}%
    \SetMainColor{#2}%
    \SetMainTextColor{#3}%
}

% Druckt die aktuelle Kopfleiste nochmal.
\newcommand{\PrintHeader}[1]{%
    \begin{textblock}{1}(0,0)%
        \begin{sideways}%
            \setlength{\fboxsep}{3pt}
            \colbox{\maincolor}{\maintextcolor}{{\Huge \textsl{\textbf{#1}}}}%
            \hspace*{1.5cm}%
        \end{sideways}%
    \end{textblock}%
}

% Makro für Monat in Fußzeile
\newcommand{\SetFusszeile}[1]{%
    \fancyfoot[RE,LO]{\textsl{SuperMAG #1}}
    \hypersetup{pdftitle={SuperMag #1}}
}

%%%%%%%%%%%%%%%%%%%%
% Links
%%%%%%%%%%%%%%%%%%%%

% neuer Zähler für Links in einem Artikel
\newcounter{linkcounter}

% Makro für Links innerhalb eines Textes
% Benutzung: \Link{URL}
\newcommand*{\Link}[2][\arabic{linkcounter}\stepcounter{linkcounter}]{~\href{#2}{[#1]}}

% Makro für den Text "Verweise"
% Benutzung: \begin{Quellen} ... \end{Quellen}
\newenvironment{Quellen}[1][\linewidth]
{%
    \noindent{}%
    \begin{minipage}{#1}
        \hfill{}%
        \begin{tabular}{r}
            \textcolor{dunkelgrau}{\normalfont\large\scshape Verweise} \\[-1.9ex]
            \textcolor{\maincolor}{\rule{2cm}{1.5pt}}\\[-0.8ex]
        \end{tabular}
        \renewcommand*{\labelenumi}{[\theenumi]}
        \begin{small}
            \begin{Aufzaehlung}
}
{%
            \end{Aufzaehlung}
        \end{small}
        \renewcommand*{\labelenumi}{\theenumi.}
    \end{minipage}
}

% Makro für Quellen am Ende eines Textes
% Benutzung: \Quelle{Bezeichung}{URL}{Linktext}
\newcommand*{\Quelle}[3]{\item #1{}: \href{#2}{#3}}

%%%%%%%%%%%%%%%%%%%%%%%%%%%%%%%%%
% Allgemeine Definitionen
%%%%%%%%%%%%%%%%%%%%%%%%%%%%%%%%%

% freiesMagazin-Schriftzug
\newcommand*{\fm}{\mbox{\textbf{\textsf{\textcolor{dunkelgrau}{freies}}}\textsf{\textcolor{orange}{Magazin}}}}

% Sanfter Zeilenumbruch
% Benutzung: \sbreak[ABSTAND]
\newcommand{\sbreak}[1][-1em]{\linebreak[4]\\[#1]}

% Spalte ausfüllen und umbrechen.
% Benutzung: \cbreakfill
\newcommand{\cbreakfill}{\vfill\columnbreak}

% Zeile sanft beenden und dann Spalte umbrechen.
% Benutzung: \csbreakfill
\newcommand{\csbreakfill}{\sbreak\cbreakfill\noindent}

%%%%%%%%%%%%%%%%%%%%%%%%%%%%%%%%%
% Textstil
%%%%%%%%%%%%%%%%%%%%%%%%%%%%%%%%%

% Umgebung für Befehlskästen
% Benutzung: \begin{Befehl}[BREITE] ... \end{Befehl}
\lstnewenvironment{Befehl}[1][1]
{\lstset{style=StyleCommand,linewidth=#1\linewidth}}
{}

% Befehle im Fließtext
\newcommand*{\term}[1]{%
    \lstinline[style=StyleListingBasic,basicstyle=\ttfamily\bfseries]|#1|%
}

% Umgebung für Listings mit Zeilennummern
% Benutzung: \begin{Listing}[BREITE] ... \end{Listing}
\lstnewenvironment{Listing}[1][1]
{\lstset{style=StyleListing,linewidth=#1\linewidth}}
{}

% Makro für Menueinträge
\newcommand*{\Menu}[1]{\textit{"`#1"'}}

% Makro für Tasten
\newcommand*{\Taste}[1]{{\small\keystroke{#1}}}

% Makro für Pakete
\newcommand*{\Paket}[1]{\textbf{\mbox{#1}}}

% Makro für Auflistungen
% Benutzung: \begin{Auflistung}
% \item Item 1
% \item Item 2
% \end{Auflistung}
\newenvironment{Auflistung}
{\begin{itemize}[itemsep=0.0em,leftmargin=*]}
{\end{itemize}}

% Makro für Aufzählungen
% Benutzung:
% \begin{Aufzaehlung}
% \item Item 1
% \item Item 2
% \end{Aufzaehlung}
\newenvironment{Aufzaehlung}
{\begin{enumerate}[itemsep=0.0em,leftmargin=*]}
{\end{enumerate}}


